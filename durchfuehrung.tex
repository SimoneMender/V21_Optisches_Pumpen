\section{Durchführung}
Als Erstes werden der Polarisationsfilter und das $\lambda/4$-Plättchen aus dem Strahlengang genommen, damit zur Justierung mehr Licht auf den Detektor trifft.
Dann wird mit den Linsen der Strahlengang so justiert, dass die Intensität am Lichtdetektor maximal ist.
Nun werden die entfernten optischen Elemente wieder in den Strahlengang gestellt und der
Aufbau wird mit einer schwarzen Decke abgedeckt, sodass kein von außen
einfallendes Licht auf den Lichtdetektor trifft.

Da die Feldstärken der Spulen in der gleichen Größenordnung liegen wie das
Erdmagnetfeld\footnote{Die horizontale Komponente beträgt in Deutschland etwa
 $20\,\upmu\text{T}$ und die vertikale etwa $44\,\upmu\text{T}$. (von Wikipedia..Quelle mache ich, wenn das wieder klappt mit git)},
 muss das Erdmagnetfeld kompensiert werden. Die horizontale Komponente
 wird durch eine parallele Ausrichtung der optischen Achse zur Nord-Süd-Richtung minimiert.
 Die vertikale Komponente des Erdmagnetfeldes wird durch die Vertikalfeldspule minimiert.
 Dazu wird das Magnetfeld der Vertikalspule solange variiert, bis der auf dem Oszilloskop zu sehende Peak minimal ist.

 Die eigentliche Messung besteht darin, die Frequenzen der Sweep-Spule von  $100\,\text{kHz}$ in $100\,\text{kHz}$-Schritten auf $1\,\text{MHz}$ zu
 erhöhen und für jede Frequenz die Resonanzpositionen auszumessen.
 Die Stärke des gesamten Horizontalfeldes setzt sich aus dem Sweep-Feld und dem Horizontalfeld zusammen, da Frequenzen von
 $300\,\text{kHz}$ zusätzlich zu der Sweep-Spule mit der Horizontalfeld-Spule ein weiteres horizontales Feld erzeugt werden muss,
 um den Bereich des oszillierenden Feldes auf die Resonanzen zu verschieben.
